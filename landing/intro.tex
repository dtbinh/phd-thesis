\section{Motivation}
One of the great challenges in computer animation is to physically
simulate a virtual character performing highly dynamic motion with
agility and grace. A wide variety of athletic movements, such as
acrobatics or freerunning (parkour), involve frequent transitions
between airborne and ground-contact phases. How to land properly to
break a fall is therefore a fundamental skill athletes must acquire. A
successful landing should minimize the risk of injury and disruption
of momentum because the quality of performance largely depends on the
athlete's ability to safely absorb the shock at landing, while
maintaining readiness for the next action. To achieve a successful
landing, the athlete must plan coordinated movements in the air,
control contacting body parts at landing, and execute fluid
follow-through motion. The basic building blocks of these motor skills
can be widely used in other sports that involve controlled falling and
rolling, such as diving, gymnastics, judo, or wrestling.

We introduce a new method to generate agile and natural human
falling and landing motions in real-time via physical simulation
without using motion capture data or pre-scripted animation (Figure
\ref{fig:landing_teaser}). We develop a general controller that allows the
character to fall from a wide range of heights and initial speeds,
continuously roll on the ground, and get back on its feet, without
inducing large stress on joints at any moment. Previous controllers
for acrobat-like motions either precisely define the sequence of
actions and contact states in a state-machine structure, or directly
track a specific motion capture sequence. Both cases fall short of
creating a generic controller capable of handling a wide variety of
initial conditions, overcoming drastic perturbations in runtime, and
exploiting unpredictable contacts.

Our method is inspired by three landing principles informally
developed in freerunning community. First, reaching the ground with
flexible arms or legs provides “cushion” time to dissipate energy over
a longer time window rather than absorbing it instantly at impact. It
also protects the important and fragile body parts, such as the head,
the pelvis, and the tailbone. Second, it is advisable to distribute
the landing impact over multiple body parts to reduce stress on any
particular joint. Third, it is crucial to utilize the friction force
generated by landing impact to steer the forward direction and control
the angular momentum for rolling, a technique referred to as
"blocking" in the freerunning community. These three principles
outline the most commonly employed landing strategy in practice:
landing with feet or hands as the first point of contact, gradually
lowering the center of mass (COM) to absorb vertical impact, and
turning a fall into a roll on the ground, with the head tightly tucked
at impact moment.

However, translating these principles to control algorithms in
a physical simulation is very challenging. During airborne, the
controller needs to plan and achieve the desired first point of
contact and the angle of attack, in the absence of control over the
characters global motion in the air. Instead of solving a large,
nonconvex two-point boundary value problem, we develop a compact
abstract model which can be simulated efficiently for real-time
applications. To strike the balance between accuracy and efficiency,
our algorithm replans the motion frequently to compensate the
approximation due to the simplicity of the model. When the character
reaches the ground, the controller needs to take a series of
coordinated actions involving active changes of contact points over
a large area of human body. Our algorithm executes three consecutive
stages, impact, rolling, and getting-up by controlling poses,
momentum, and contacts at key moments. Furthermore, the airborne and
landing phases are interrelated and cannot be considered in
isolation: the condition for a successful landing defines the
control goals for the airborne phase while the actions taken during
airborne directly impact the landing motion. We approach this
problem in a reverse order of the action sequence: designing a
robust landing controller, deriving a successful landing condition
from this controller, and developing an airborne controller to achieve
the landing condition.


We demonstrate that our control algorithm is general,
efficient, and robust. We apply our algorithm to a variety of
initial conditions with different falling heights, orientations, and
linear and angular velocities. Because the motion is simulated in
real-time, users can apply perturbation forces to alter the course
of the character in the air. Our algorithm is able to efficiently
update the plan for landing given the new situations. We also
demonstrate different strategies to absorb impact, such as a dive
roll, a forward roll, or tumbling. The same control algorithm can be
applied to characters with very different body structures and mass
distributions. We show that a character with unusual body shape can
land and roll successfully.  
Finally, our experiments empirically showed that the algorithm induces
smaller joint stress, except for the contacting end-effectors. In
the worst case of our experiments, the average joint stress is still
four times lower than landing as a passive ragdoll.



