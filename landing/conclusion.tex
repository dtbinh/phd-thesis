\section {Discussion}

We introduced a real-time physics-based technique to simulate
strategic falling and landing motions. 
Our control algorithm reduces
joint stress due to landing impact and allows the character to
efficiently recover from the fall.
 Given an arbitrary initial position
and velocity in the air, our control algorithm determines an
appropriate landing strategy and an optimal sequence of actions to
achieve the desired landing velocity and angle of attack. The
character utilizes virtual forces and joint-tracking control
mechanisms during the landing phase to successfully turn a fall into a
roll. We demonstrated that our control algorithm is general,
efficient, and robust by simulating motions from different initial
conditions, characters with different body shapes, different physical
environments, and scenarios with real-time user perturbations.  The
algorithm guides the character to land safely without introducing the
large stress at every joint except for the contacting end-effectors.

Freerunning is a great exemplar to demonstrate human athletic
skills. Those wonderfully simple yet creative movements provide a
rich domain for future research directions. Based on the contribution
of this work, we would like to explore other highly dynamic skills in
freerunning, such as cat crawl, underbar, or turn vault. These motions
are extremely interesting and challenging to simulate because they
involve sophisticated planning and control in both cognitive and motor
control levels, as well as complex interplay between the performer and
the environment.

The landing strategies described in this work are suitable for highly
dynamic activities, but not optimal for low-clearance falls from
standing height. There is a vast body of research work in biomechanics
and kinesiology studying fall mechanics of human from standing
height. One future direction of interest is to integrate this domain
knowledge with physical simulation tools to explore new methods for
fall prevention and protection.

