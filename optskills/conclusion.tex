\section{Conclusion} 
\label{sec:optskills_conclusion}

We presented a new evolutionary optimization algorithm for
  learning parameterized dynamic motor skills.
  Instead of individually acquiring optimal policies for each task,
  our algorithm simultaneously learns the policies for the entire range of
  tasks.
  The key insight of our algorithm is to sample in the space of policy
  parameters rather than directly sample in the high-dimensional space
  of parameterized skill function parameters. Since the solution in
  a parameterized problem is a curve segment rather than a point,
  our approach maintains a parameterized probability distribution along the
  mean segment and evolves it using selected elite samples.
  We demonstrated that our algorithm shows faster convergence when
  comparing to the baseline algorithm, CMA-ES, especially when using a
  cubic parameterized skill function.
  %% Since we learned an explicit mapping from a task parameter to policy
  %% parameters, the trained model will produce better control parameters for
  %% unseen tasks than the model from individual learning approach.}

  Although our algorithm optimizes parameterized tasks automatically, it is the user's
  responsibility to set a feasible task range achievable by the given
  parameterization of the control policy. If the range is too wide,
  the optimization will not converge to a good solution due to the
  intrinsic limitations of the space of policy parameters. 
% One
%   potential remedy is to simultaneously adapt the task
%   range when fitting the mean segment.

  For future consideration, we plan on extending the task
  interpolation parameters into higher dimensions, which may afford
  greater flexibility of the resultant motor skills considerably.  For
  example, currently we parameterize the jump controller to act in the
  vertical direction only, which obviates its use in general
  navigation applications. A consequence of increasing the dimension
  of the task parameter is that it will require the mean function be
  extended to the hyperplane in this new parameter space instead of a
  segment.
