\section{Conclusion}
We present a novel framework for controller design using human
coaching and learning techniques. The framework takes in a blackbox
controller and improves it through an iterative learning process of
coaching and practicing. The user can directly give high-level
instructions to the virtual character using control rigs while the
character can efficiently optimize its motor skill using a new
sampling-based optimization method, CMA-C. Once a controller is
developed, parameterizing it to a family of similar controllers for
concatenation can be done without additional effort from the user.

In this work, we demonstrated that developing complex motor
controllers does not need any motion trajectories. However, if the
user wishes to utilize motion examples to train the virtual character,
instead of verbal instructions, our framework can be adapted by
including a control rig that modulates the reference trajectories,
similar to the sampling approach proposed by Liu \etal
\cite{Liu:2010:SCM,Liu:2012:TRC}. 

Our current implementation requires the user to construct
instructions as a script according to the template grammars. One
possible future direction is to augment our framework with a natural
language processor so the user can use more colloquial
commands to train the character, such as ``Lower your body a bit
more'', instead of ``MOVE COM down BY 0.2m''. In addition, we would
like to explore Kinet-like sensors to enable the possibility of ``teaching by demonstration''.
Therefore, one possible future direction is to augment our framework
with two different types of interfaces: a natural language processor
and a Kinect-like sensor. These two interfaces will allow the user to
train the character by describing the motion in human language while
demonstrating the movement using his/her own body.
