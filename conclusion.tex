%1234567890123456789012345678901234567890123456789012345678901234567890123456789
\chapter{Conclusion}

Within this dissertation, we have presented methods for developing agile motor
skills on virtual and real humanoids.
The dissertation started from designing controllers for the specific tasks
such as falling strategies,
and covered a set of general computational tools to develop
various motor skills.
Using the proposed techniques, various motor skills can be intuitively designed
in virtual simulation and easily transferred to physical systems.
Several optimization algorithms are also designed to develop more versatile and
robust physics-based controllers for humanoids within short amount of time.
Further, falling of virtual and real humanoids are extensively studied to
ensure safety of humanoids and to achieve smooth transitions between motor
skills.

Chapter 3 described a method to generate agile falling and landing motions of
virtual characters in real-time via physical simulation
without using motion capture data or pre-scripted animation.
By designing novel controllers based on three landing principles informally
developed in Parkour community, we can develop a general controller that
allows the character to fall from a wide range of heights and initial speeds,
roll on the ground, and get back on its feet, without inducing large stress on
joints at any moment.

Chapter 4 introduced a new planning algorithm to minimize the damage
of humanoid falls by utilizing multiple contact points.
Instead of selecting among a collection of manually designed control
strategies, we propose a novel algorithm which plans for appropriate
falling motions to a wide variety of falls.
Our algorithm covers various falling strategies from a single step to recover a
gentle nudge, to a rolling motion to break a high-speed fall.

The iterative learning framework of Chapter 5 allows users to intuitively
develop dynamic controllers for virtual characters only using high-level,
human-readable instructions.
The implementation of instructions uses ``control rigs'', which is an
intermediate layer of control module 
for facilitating mapping between high-level instructions and manipulating
multiple low-level control variables.
The control rigs are design for utilizing the human coach's knowledge to reduce 
the search space for control optimization.

The optimization problems formulated in Chapter 5 can be efficiently solved
using a new sampling-based optimization method, Covariance Matrix
Adaptation with Classification (CMA-C), explained in Chapter 6.
Based on the human ability to learn from failure,
CMA-C utilizes the failed simulation samples to approximate infeasible regions
parameters, resulting a faster convergence for the CMA-ES optimization.

Chapter 7 shows how to optimize a parameterized motor skill which
is essential for autonomous robots operating in an unpredictable environment.
Our algorithm achieves the faster convergence rate by evolving a parameterized
probability distribution for the entire range of tasks.

The method described in Chapter 8 provide us a framework for reducing hardware
experiments which is usually very time-consuming and costly.
The goal of this method is to learn the difference between simulation and
hardware systems, and optimize control policy that works on the target system.
Instead of learning the model from scratch, our method learns only the
difference between a simulation model and hardware, which results the reduced
numbers of hardware experiments.

In conclusion, this dissertation proposed a set of algorithms and framework
that can be important steps toward the development of agile motions on virtual
and real humanoids.
However, the entire pipeline is yet exhaustively tested for real robots.
For instance, the proposed iterative learning framework of Chapter 5 is only
tested on a virtual character and not verified on a real humanoid.
Because a robot has different body and joint structures,
a user may have a difficulty to teach a robot with previously
designed high-level instructions and control rigs.
An interesting future direction is to resolve these difficulty issues by 
revising the design of framework components to expedite the direct 
communication between a user and a real robot.

We previously demonstrated a framework for learning dynamic bias
a simple legged robot in Chapter 8, but it is not fully verified on a
full-scale humanoid.
Because humanoid robots usually have much higher degrees of freedom ranging
from 16 to 34 [GP, Atlas, Sarcos], a na\"{\i}ve approach will require
tremendous amount of data which is infeasible to be obtained.
There are several possible approaches for resolving this curse of
dimensionality, such as projecting into a lower dimensional space using a
simplified model or adopting active learning to maximize the amount of
information from a single hardware experiment.












